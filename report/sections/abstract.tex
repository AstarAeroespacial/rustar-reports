This professional project aimed to develop RUSTAR, an open-source framework for the ground segment of satellite missions. The development was carried out within the context of the Astar project at the Faculty of Engineering of the University of Buenos Aires, which focuses on developing \gls{cubesat} satellites, although the system is designed to be easily extensible to any satellite type. RUSTAR enables optimal software operation of ground stations dedicated to \gls{telemetrytrackingandcontrol} (\gls{ttc}) tasks, facilitating satellite communications through \gls{softwaredefinedradio} (\gls{SDR}) devices and implementing the necessary modulation and demodulation schemes for signal processing. The system incorporates real-time satellite tracking capabilities, allowing the robotic arm's rotation system that controls the antenna to accurately point at the satellite during its orbital pass.

To ensure secure, efficient, and reliable data exchange, an appropriate communication protocol was implemented. The system is remotely accessible, accommodating scenarios where operators and the ground station may be located at different sites. A graphical user interface was developed to facilitate mission status visualization and command transmission. The framework was designed with a modular and extensible approach, built entirely on open-source tools, enabling its adaptation to different configurations and requirements across various satellite missions.

The purpose of this work was to provide a comprehensive and reusable platform that optimizes operator workflows and facilitates satellite interaction across different missions. Furthermore, being linked to the academic environment and open-source in nature, the framework contributes to strengthening practical training in the aerospace field, serving as a learning and experimentation tool for students, researchers at the university, and the broader scientific community.