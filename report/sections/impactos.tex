\subsection{Impacto social}

El desarrollo de este proyecto busca contribuir al fortalecimiento del ecosistema espacial argentino desde una pespectiva abierta, accesible y vinculada a la formación universitaria. La construcción de un \textit{framework} de código abierto para el segmento terreno permite reducir la barrera de entrada al desarrollo y operación de misiones satelitales. También promueve la interoperabilidad y colaboración entre distintos proyectos y misiones, al ofrecer una infraestructura común del segmento terreno.

La naturaleza del sistema permite que cualquier institución (escuelas, clubes, grupos de aficionados) pueda desplegar su propia estación terrena y formar parte del segmento terreno de distintas misiones. Esta accesibilidad tiene el potencial de democratizar el acceso al espacio, mediante una solución elaborada en el seno del sistema universitario argentino, con la ambición de atraer y contribuir a la formación de los futuros talentos del sector aeroespacial.

Asimismo, el sistema puede resultar de utilidad para institutos o grupos de investigación, al contar con una infraestructura adaptable a las necesidades que puedan tener, como por ejemplo adquisición, análisis y distribución de datos provenientes de misiones satelitales.

El proyecto aporta al desarrollo de soberanía tecnológica y busca contribuir al posicionamiento de Argentina en la nueva era de actividades espaciales que se está abriendo.

\subsection{Impacto ambiental}

Desde el punto de vista ambiental, el impacto directo del sistema es acotado, al tratarse exclusivamente de software.

No obstante, este sistema puede tener un impacto ambiental positivo de forma indirecta, ya que podría servir de base para de misiones satelitales con objetivos ambientales, como el monitoreo climático, la detección temprana de incendios, la gestión de recursos naturales o la optimización de procesos que se beneficien de información recolectada por una misión satelital.