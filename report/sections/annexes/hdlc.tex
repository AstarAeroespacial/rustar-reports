El protocolo \gls{HDLC} (High-Level Data Link Control) es un protocolo de capa de enlace de datos ampliamente utilizado en comunicaciones satelitales, especialmente en misiones de órbita baja y proyectos de CubeSats. Constituye el estándar de facto para el intercambio de información entre estaciones terrenas y satélites de radioaficionados debido a su robustez, eficiencia y capacidad de operar en entornos con alta tasa de errores.

\subsubsection{Estructura de frame}

El protocolo \gls{HDLC} organiza los datos en frames delimitados por secuencias de sincronización denominadas \textit{flags}. Estos \textit{flags} marcan el inicio y fin de cada paquete, permitiendo al receptor identificar los límites de las unidades de información en el flujo continuo de bits.

\begin{table}[h]
\centering
\begin{tabular}{|c|c|c|c|c|c|}
\hline
\textbf{\textit{Flag}} & \textbf{Address} & \textbf{Control} & \textbf{Info.} & \textbf{FCS} & \textbf{\textit{Flag}} \\
\hline
01111110 & 8 bits & 8 bits & * & 16 bits & 01111110 \\
\hline
\end{tabular}
\caption{Estructura del frame HDLC}
\label{tab:hdlc-frame}
\end{table}

Cada frame contiene los siguientes campos:

\begin{itemize}
    \item \textbf{\textit{Flag} de apertura}: Secuencia de sincronización que marca el inicio del frame (típicamente el patrón binario \texttt{01111110}).
    \item \textbf{Dirección}: Campo de dirección de 8 bits que identifica la estación destinataria del frame. En configuraciones punto a punto, este campo puede utilizarse con un valor fijo.
    \item \textbf{Control}: Campo de control de 8 bits que especifica el tipo de frame (información, supervisión o no numerado) y contiene números de secuencia para el control de flujo y la detección de frames duplicados o perdidos.
    \item \textbf{Campo de información}: Contiene la carga útil de datos, que puede ser telemetría recibida del satélite o comandos a transmitir. Su longitud es variable.
    \item \textbf{Secuencia de verificación de frame (FCS)}: Campo de 16 bits que contiene un código de redundancia cíclica (CRC-16) calculado sobre el contenido del frame, permitiendo detectar errores de transmisión.
    \item \textbf{\textit{Flag} de cierre}: Secuencia de sincronización que marca el fin del frame (idéntica al \textit{flag} de apertura).
\end{itemize}

El mecanismo de FCS garantiza que los datos recibidos correspondan exactamente a los transmitidos. Cuando se detecta una discrepancia en el CRC, el frame se descarta como corrupto, permitiendo al sistema de capas superiores implementar mecanismos de retransmisión cuando sea necesario.

\subsubsection{Transparencia: Bit stuffing y destuffing}

Un desafío fundamental en el diseño de \gls{HDLC} es evitar que el contenido del frame se confunda con las secuencias de sincronización (\textit{flags}). Para resolver este problema, el protocolo emplea una técnica denominada \textit{bit stuffing}.

El proceso de bit stuffing opera de la siguiente manera:

\begin{itemize}
    \item Durante la transmisión, cuando en el flujo de datos aparecen cinco bits consecutivos en 1, el \textit{framer} inserta automáticamente un 0 adicional después de la quinta posición.
    \item Durante la recepción, el \textit{deframer} realiza el proceso inverso (bit destuffing), eliminando estos ceros insertados para recuperar la secuencia original de datos.
    \item Esta técnica garantiza que la única secuencia de seis unos consecutivos que puede aparecer en la transmisión es la correspondiente a los \textit{flags} de sincronización.
\end{itemize}

Esta técnica asegura la transparencia del protocolo, es decir, la capacidad de transmitir cualquier patrón arbitrario de bits sin ambigüedades respecto a los delimitadores de frame. El contenido de información puede contener cualquier secuencia de bits sin interferir con el mecanismo de sincronización.

\subsubsection{Optimización: \textit{Flags} compartidos}

El protocolo \gls{HDLC} soporta el concepto de \textit{flags} compartidos, una optimización que mejora la eficiencia en transmisiones continuas de múltiples paquetes. En este esquema, el \textit{flag} de cierre de un frame puede actuar simultáneamente como \textit{flag} de apertura del siguiente frame.

Esta característica optimiza el uso del ancho de banda al eliminar la necesidad de transmitir \textit{flags} redundantes entre frames consecutivos, reduciendo el overhead del protocolo en comunicaciones de alta velocidad donde se transmiten múltiples paquetes de forma continua.

\subsubsection{Operación robusta en canales adversos}

El diseño del protocolo \gls{HDLC} lo hace especialmente adecuado para operar en condiciones adversas del canal de comunicación, características típicas de los enlaces satelitales:

\begin{itemize}
    \item \textbf{Resincronización}: El receptor busca patrones de \textit{flag} de forma incremental en el flujo de bits entrante, permitiendo recuperar la sincronización incluso después de períodos de pérdida de señal.
    \item \textbf{Descarte de basura}: Cuando se detectan datos inválidos entre frames (basura o ruido del canal), el protocolo los descarta automáticamente y continúa buscando el siguiente \textit{flag} válido.
    \item \textbf{Validación por CRC}: Solo los frames con FCS válido son aceptados y entregados a las capas superiores, garantizando la integridad de los datos en presencia de errores de transmisión.
    \item \textbf{Buffer circular}: Los implementaciones típicas utilizan un búfer circular que acumula bits entrantes, permitiendo procesar el flujo de forma continua sin pérdida de información.
\end{itemize}

Estas características hacen de \gls{HDLC} un protocolo robusto y confiable para comunicaciones satelitales, capaz de mantener la integridad de los datos incluso en presencia de alta tasa de errores, desvanecimientos y condiciones variables del enlace de radiofrecuencia.
