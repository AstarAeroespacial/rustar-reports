\subsection{Tracking}

El módulo de Tracking proporciona las capacidades necesarias para operar la estación terrena con satélites a partir de su \hyperref[gls:TLE]{TLE} (Two-Line Element). Sus funciones principales son:
\begin{itemize}
\item Calcular la posición y velocidad del satélite en un instante dado.
\item Obtener acimut y elevación para orientar la antena y seguir al satélite durante un pase.
\item Predecir la próxima ventana de visibilidad para una estación y satelite concretos.
\item Calcular la corrección por efecto Doppler para ajustar las frecuencias de transmisión y recepción.
\end{itemize}
Esta funcionalidad fue implementada como un módulo de \textit{Tracking} que se encarga de las funcionalidades mencionadas.

Utilizamos dos librerías: \hyperref[gls:SGP4]{\texttt{sgp4}} y \texttt{predict-rs}.
Estas librerías utilizan \hyperref[gls:SGP4]{SGP4} (\textit{Simplified General Perturbations 4}), que es el propagador orbital estándar usado para convertir un \hyperref[gls:TLE]{TLE} en la posición y velocidad del satélite en un momento dado.

Dado un \hyperref[gls:TLE]{TLE} y un instante, el módulo calcula la posición y la velocidad del satélite. Conociendo además la posición de la estación terrena (latitud, longitud y altitud), se determinan el acimut y la elevación necesarios para que la antena siga al satélite en cada instante.

El módulo incluye también una función para calcular el próximo \textit{pass}, es decir, la siguiente ventana de visibilidad del satélite respecto a una estación concreta, con los instantes de inicio y fin de comunicación previstos.

Además, se proporciona la corrección por efecto Doppler, calculada en Hz, que permite ajustar las frecuencias de transmisión y recepción de la estación para compensar el desplazamiento de frecuencia durante la ventana de visibilidad.

La corrección por efecto Doppler se calcula con la siguiente aproximación (convención: $\dot{\rho}>0$ si el satélite se aleja):
\[
	{ \textbf{Downlink (recepción en estación):}\quad }
  f_{\mathrm{rx}} \approx f_{\mathrm{tx,sat}}\!\left(1 - \frac{\dot{\rho}}{c}\right)
\]
\[
	{ \textbf{Uplink (transmisión desde estación):}\quad }
  f_{\mathrm{tx,gs}} \approx f_{\mathrm{rx,sat}}\!\left(1 + \frac{\dot{\rho}}{c}\right)
\]

donde:
\begin{itemize}
  \item $f_{\mathrm{tx,sat}}$ — Frecuencia que transmite el satélite (downlink).
  \item $f_{\mathrm{rx}}$ — Frecuencia que debe sintonizar la estación (downlink).
  \item $f_{\mathrm{rx,sat}}$ — Frecuencia que espera recibir el satélite (uplink).
  \item $f_{\mathrm{tx,gs}}$ — Frecuencia a la que debe transmitir la estación (uplink).
  \item $\dot{\rho}$ — Velocidad radial en m/s (positivo = alejándose, negativo = acercándose).
  \item $c$ — Velocidad de la luz.
\end{itemize}

\subsubsection{Validación}
Para poder validar que los cálculos de acimut y elevación eran correctos, comparamos nuestros valores con los que provee \textit{Orbitron} e intentamos que la diferencia entre los valores sea mínima.

\textbf{Orbitron}: es un software de seguimiento satelital para Windows que, a partir de \hyperref[gls:TLE]{TLE}s, calcula posiciones, acimut/elevación y ventanas de visibilidad usando propagación \hyperref[gls:SGP4]{SGP4} y modelos geométricos estándar.

En cuanto al \textit{Doppler}, comparamos los resultados de nuestros cálculos con una biblioteca de Python llamada \textit{skyfield}.

\textbf{Skyfield}: es una biblioteca de Python para astronomía y mecánica celeste que permite calcular posiciones y velocidades con alta precisión. Para \hyperref[gls:TLE]{TLE}s, integra un propagador compatible con \hyperref[gls:SGP4]{SGP4} y provee utilidades para transformaciones de marcos y cálculo de correcciones como el efecto Doppler.

\begin{figure}[ht]
  \centering
  \includegraphics[width=0.85\linewidth]{images/validacion_doppler.png}
  \caption{Comparación de la corrección Doppler calculada por el módulo y la referencia (Skyfield).}
  \label{fig:validacion_doppler}
\end{figure}