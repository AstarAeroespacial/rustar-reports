El esquema actual de la base de datos es el siguiente:

\begin{figure}[h]
    \centering
    \includegraphics[width=1.0\linewidth]{images/bdd_schema.png}
    \caption{Esquema de la base de datos de RUSTAR}
\end{figure}

El esquema modela los elementos principales del sistema. La tabla \texttt{ground\_stations} registra las estaciones terrenas con su identificador y posición geográfica (latitud, longitud y altitud). La tabla \texttt{satellites} conserva los satélites disponibles junto con su nombre, el TLE vigente y las frecuencias de comunicación. Ambas tablas se enlazan con las distintas misiones programadas en la tabla \texttt{jobs}, que representa una asignación de seguimiento entre una estación y un satélite dentro de un intervalo temporal definido.

Cada \texttt{job} puede incluir múltiples comandos a ejecutar sobre la estación, almacenados en \texttt{job\_commands}. La ejecución de una misión se monitorea mediante \texttt{jobs\_status\_updates}, donde se conservan los cambios de estado más recientes asociados al identificador del job. Finalmente, la tabla \texttt{telemetry} archiva los paquetes de telemetría recibidos, manteniendo la referencia al satélite y a la estación que capturó el dato.

Las claves foráneas garantizan la integridad referencial entre las entidades: los jobs sólo pueden asociarse a satélites y estaciones existentes, los comandos y actualizaciones de estado dependen de un job válido, y los registros de telemetría se vinculan con las mismas entidades que participaron en la captura. Este diseño permite consultar el historial de operaciones y telemetría de forma consistente y auditable.
