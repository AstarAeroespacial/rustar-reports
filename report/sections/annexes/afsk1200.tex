\subsection{AFSK1200}
\label{sec:afsk1200}

La solución implementada utiliza AFSK1200 (\textit{Audio Frequency-Shift Keying} a 1200 baud), un esquema de modulación ampliamente empleado en comunicaciones satelitales de baja velocidad, como en protocolos tipo AX.25 o misiones educativas. No se ahondará en detalles teóricos de modulación y demodulación ya que escapa al alcance de este documento.

Los esquemas de modulación y demodulación se implementaron en \gls{grc}, que permite implementar de forma gráfica \textit{flowgraphs} de procesamiento de señales. Luego, se pueden exportar a programas en Python, utilizados por la aplicación de la estación terrena para realizar la modulación y demodulación.

A continuación se muestran los flowgraphs de \gls{grc} utilizados durante el desarrollo de este trabajo.

\begin{figure}[H]
  \centering
  \includegraphics[width=0.95\linewidth]{images/afsk1200_mod.png}
  \caption{Flowgraph de \gls{grc} para modulación AFSK1200.}
  \label{fig:afsk1200_mod}
\end{figure}

\begin{figure}[H]
  \centering
  \includegraphics[width=0.95\linewidth]{images/afsk1200_demod.png}
  \caption{Flowgraph de \gls{grc} para demodulación AFSK1200.}
  \label{fig:afsk1200_demod}
\end{figure}

Para generar los programas en Python utilizables por la aplicación de la estación terrena, es necesario configurar, en el bloque "Options" la opción "Generate Options" en "No GUI" y deshabilitar los bloques gráficos. Luego utilizar la opción "Generate the flow graph".