La API expone los servicios que permiten operar el sistema desde aplicaciones externas. Cada conjunto de endpoints agrupa responsabilidades específicas para administrar satélites, estaciones, telemetría, planificación de pases y trabajos de seguimiento.

\paragraph{Gestión de satélites}

Estos endpoints permiten mantener el catálogo de satélites y sus parámetros orbitales, así como consultar los comandos disponibles para operar cada plataforma.

\begin{itemize}
    \item \texttt{GET /api/satellites}: Obtener todos los satélites con sus \gls{tle}.
    \item \texttt{GET /api/satellites/\{id\}}: Obtener un satélite específico a partir de su identificador.
    \item \texttt{PUT /api/satellites/\{id\}}: Actualizar el \gls{tle} registrado para un satélite.
    \item \texttt{GET /api/satellite/\{id\}/commands}: Consultar los comandos disponibles para el satélite indicado.
\end{itemize}

\paragraph{Gestión de estaciones terrenas}

Provee operaciones para registrar estaciones terrenas, consultarlas y asignar el satélite que deben seguir en un momento dado.

\begin{itemize}
    \item \texttt{GET /api/ground-stations}: Obtener todas las estaciones terrenas registradas.
    \item \texttt{GET /api/ground-stations/\{id\}}: Obtener una estación terrena específica por su identificador.
    \item \texttt{POST /api/ground-stations}: Crear una nueva estación terrena.
    \item \texttt{PUT /api/ground-stations/\{id\}/satellite}: Actualizar el satélite que la estación debe seguir.
\end{itemize}

\paragraph{Telemetría}

La API permite consultar la telemetría decodificada con paginación y administrar el decodificador asociado a cada satélite para ajustar el procesamiento de los paquetes recibidos.

\begin{itemize}
    \item \texttt{GET /api/satellite/\{id\}/telemetry}: Obtener los paquetes de telemetría decodificados de un satélite (parámetros \texttt{pageSize} y \texttt{pageNumber}).
    \item \texttt{GET /api/satellite/\{id\}/telemetry/decoder}: Consultar la configuración del decodificador de telemetría del satélite.
    \item \texttt{PUT /api/satellite/\{id\}/telemetry/decoder}: Actualizar la configuración del decodificador del satélite.
\end{itemize}

\paragraph{Seguimiento}

Estos recursos utilizan el módulo de Tracking (descrito en detalle en el Anexo~\ref{sec:tracking}) para calcular oportunidades de observación futuras, tanto desde la perspectiva de un satélite como de una estación terrena. A partir de los parámetros orbitales (TLE) y la ubicación de las estaciones, se predicen las ventanas de visibilidad con sus tiempos de adquisición y pérdida de señal, permitiendo filtrar por satélites o estaciones terrenas de interés.

\begin{itemize}
    \item \texttt{GET /api/satellites/\{id\}/passes}: Obtener los próximos pases del satélite sobre las estaciones terrenas disponibles.
    \item \texttt{GET /api/ground-stations/\{id\}/passes}: Obtener los próximos satélites que la estación podrá observar.
    \item \texttt{POST /api/ground-stations/\{id\}/passes}: Calcular los próximos satélites a observar para la estación considerando la lista \texttt{sat\_ids}.
\end{itemize}

\paragraph{Tareas de observación}

Los endpoints de trabajos encapsulan la creación y seguimiento de tareas de rastreo, incluyendo la programación de comandos que se ejecutarán sobre la estación terrena.

\begin{itemize}
    \item \texttt{POST /api/jobs}: Crear un nuevo trabajo indicando \texttt{gs\_id}, \texttt{sat\_id} y la lista de comandos.
    \item \texttt{GET /api/jobs}: Listar todos los trabajos registrados.
    \item \texttt{GET /api/jobs/\{id\}}: Obtener la información de un trabajo específico.
    \item \texttt{GET /jobs/\{id\}/status}: Consultar el estado actual del trabajo.
\end{itemize}