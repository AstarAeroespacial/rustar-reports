% Solución implementada

\subsection{Arquitectura general}

La arquitectura clásica del segmento terreno incluye dos componentes principales: el centro de control de misión y una o varias estaciones terrenas. La solución desarrollada se adapta a esta división, separando las tareas en distintos sistemas con responsabilidades bien definidas e interfaces de comunicación transparentes.

La estación terrena es el componente encargado de establecer y mantener la comunicación directa con el satélite mediante radioenlace durante las ventanas de observación. Las estaciones terrenas se instalan en sitios que podrían ser remotos o con conectividad limitada, lo que demanda un diseño autónomo, confiable y eficiente en cuanto a la gestión de recursos.

El centro de control de misión concentra las funciones de supervisión, planificación y operación. Permnite a los operadores tener una visión integrada del estado en tiempo real de las estaciones y satélites, visualización de la telemetría recibida y herramientas para planificar observaciones y enviar telecomandos a los satélites.

La comunicación entre estaciones terrenas y servidores se realiza a través de una cola de mensajería accesible por todos los nodos del sistema. Este esquema facilita la integración de múltiples estaciones, centros de control y cualquier otro tipo de aplicación, permitiendo topologías más complejas que se adapten a todo tipo de necesidades.

En conjunto, la solución propuesta se concibe como un ecosistema modular de componentes interoperables, donde las interfaces entre estos son transparentes y estandarizadas. Esta característica central convierte al sistema en un verdadero framework para la implementación del segmento terreno, adaptable a distintos contextos y escalas, desde simulaciones virtuales hasta redes distribuidas de seguimiento satelital.

A continuación, describimos los componentes esenciales del sistema que fueron desarrollados como parte de la solución presentada.
