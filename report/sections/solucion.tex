% Solución implementada

\subsection{Arquitectura general}

La arquitectura clásica del segmento terreno incluye dos componentes principales: el centro de control de misión y una o varias estaciones terrenas. La solución desarrollada se adapta a esta división, separando las tareas en distintos sistemas con responsabilidades bien definidas e interfaces de comunicación transparentes.

La estación terrena es el componente encargado de establecer y mantener la comunicación directa con el satélite mediante radioenlace durante las ventanas de observación. Las estaciones terrenas se instalan en sitios que podrían ser remotos o con conectividad limitada, lo que demanda un diseño autónomo, confiable y eficiente en cuanto a la gestión de recursos.

El centro de control de misión concentra las funciones de supervisión, planificación y operación. Permite a los operadores tener una visión integrada del estado en tiempo real de las estaciones y satélites, visualización de la telemetría recibida y herramientas para planificar observaciones y enviar telecomandos a los satélites.

La comunicación entre estaciones terrenas y servidores se realiza a través de una cola de mensajería accesible por todos los nodos del sistema. Este esquema facilita la integración de múltiples estaciones, centros de control y cualquier otro tipo de aplicación, permitiendo topologías más complejas que se adapten a todo tipo de necesidades.

En conjunto, la solución propuesta se concibe como un ecosistema modular de componentes interoperables, donde las interfaces entre estos son transparentes y estandarizadas. Esta característica central convierte al sistema en un verdadero framework para la implementación del segmento terreno, adaptable a distintos contextos y escalas, desde simulaciones virtuales hasta redes distribuidas de seguimiento satelital.

A continuación, describimos los componentes esenciales del sistema que fueron desarrollados como parte de la solución presentada.

\subsection{Estación terrena}

Las funciones que lleva a cabo para cumplir con su propósito son:
\begin{itemize}
    \item Comunicarse con el dispositivo SDR, encargado de la transmisión y recepción de señales de radiofrecuencia.
    \item Realizar la demodulación y modulación necesarios para convertir las señales en tramas de bits y viceversa.
    \item Implementar el protocolo de enlace de datos requerido para el intercambio de información.
    \item Controlar la orientación de la antena para seguir la trayectoria del satélite.
\end{itemize}

Estas funciones se ejecutan durante una ventana de observación de un satélite con el cual se quiere establecer contacto. Para programar un contacto u observación, el mecanismo provisto por la estación terrena es mediante el envío de una tarea de observación.

\subsubsection{Tareas de observación}

Las tareas de observación se envían a las estaciones terrenas a través de la cola de mensajería.

Cada tarea de observación contiene la información necesaria para llevar a cabo la operación:
\begin{itemize}
    \item Identificador del satélite, necesario para correlacionar la telemetría e información obtenida.
    \item Los parámetros orbitales (TLE) para calcular la posición y velocidad relativa del satélite.
    \item Las frecuencias de transmisión y recepción para configurar el radioenlace.
    \item La ventana temporal de observación, con la adquisición y pérdida de señal.
    \item Opcionalmente, información para transmitir al satélite en caso de ser necesario.
\end{itemize}

Una vez recibida la tarea de observación, la estación programará su ejecución para que comience automáticamente al iniciar la ventana de contacto. Además reporta el estado de la tarea mediante la cola de mensajería.

Durante la ejecución de la observación, se obtienen los datos transmitidos por el satélite y se envían a la cola de mensajería. Si hubiere, se envía al satélite la información contenida en la tarea de observación, asegurando su recepción.

\subsubsection{Radio definida por software}

La aplicación de la estación terrena se comunica directamente con el dispositivo SDR para configurar los parámetros operativos del enlace, principalmente las frecuencias de transmisión y recepción, el ancho de banda y la tasa de muestreo. Durante la recepción, el SDR entrega un flujo continuo de muestras crudas que representan la señal capturada en el dominio digital, las cuales son procesadas por las etapas de demodulación. De forma análoga, en transmisión, la aplicación genera las muestras correspondientes a la señal modulada y las envía al SDR para su conversión y emisión en radiofrecuencia.

La comunicación con el SDR se realiza a través de una capa de abstracción que define una interfaz común para los distintos dispositivos. Esto permite utilizar diferentes dispositivos SDR o fuentes de muestras sin modificar la lógica principal del sistema. Gracias a esta separación, la estación puede adaptarse a diversos entornos de hardware y esquemas de modulación, manteniendo una arquitectura flexible y fácilmente extensible.
