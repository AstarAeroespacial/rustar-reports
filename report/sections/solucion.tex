% Solución implementada
\section{Solución implementada}

\subsection{Servidor}

Como parte de la solución, se implementó un servidor con una REST API que funciona como interfaz entre el usuario y las estaciones terrenas. Entre las funciones del servidor se encuentran:

\begin{itemize}
    \item Control remoto de la estación terrena mediante la asignación de tareas de seguimiento.
    \item Recolección de datos de telemetría provenientes del satélite.
    \item Llevar registro de las distintas estaciones terrenas y trabajos asignados. 
\end{itemize}

\subsubsection{Comunicación con Estaciones Terrenas}

El servidor tiene un hilo de ejecución dedicado a manejar la comunicación con las estaciones terrenas por medio del broker MQTT. Por medio del broker el servidor puede enviarle jobs a las estaciones terrenas, además de recibir información de control y telemetría de las mismas.

El hilo se encuentra continuamente escuchando eventos MQTT del broker. Cuando recibe un evento de mensaje publicado, extrae el contenido del mensaje y el tópico al que fue publicado. En el nombre del tópico se encuentra la estación terrena del cual provino el mensaje. Con esta información, el mensaje es almacenado en una base de datos para su uso posterior. Cuando el servidor necesita acceder a los mensajes, lo hace directamente por medio de la base de datos.

Para el envío de mensajes, el hilo principal del servidor tiene una copia del cliente que puede usar para enviar mensajes a una estación terrena en particular, especificada por medio del tópico. La librería que utilizamos requiere hacer \textit{polling} constante del cliente para enviar mensajes, el hilo recibidor se encarga de eso.

El broker de MQTT está configurado para garantizar la entrega de mensajes por medio de la \textit{calidad de servicio} (QoS). Utilizamos una calidad de servicio de \textit{exactamente una vez} para asegurarnos de que cada mensaje sea entregado sin duplicación a su destinatario. El broker también puede ser configurado para manejar la autenticación de las estaciones terrenas y el servidor por TLS.

\subsubsection{REST API}

La API expone los servicios que permiten operar el sistema desde aplicaciones externas. Cada conjunto de endpoints agrupa responsabilidades específicas para administrar satélites, estaciones, telemetría, planificación de pases y trabajos de seguimiento.

\paragraph{Gestión de Satélites}

Estos endpoints permiten mantener el catálogo de satélites y sus parámetros orbitales, así como consultar los comandos disponibles para operar cada plataforma.

\begin{itemize}
    \item \texttt{GET /api/satellites}: Obtener todos los satélites con sus datos TLE.
    \item \texttt{GET /api/satellites/\{id\}}: Obtener un satélite específico a partir de su identificador.
    \item \texttt{PUT /api/satellites/\{id\}}: Actualizar el TLE registrado para un satélite.
    \item \texttt{GET /api/satellite/\{id\}/commands}: Consultar los comandos disponibles para el satélite indicado.
\end{itemize}

\paragraph{Gestión de Estaciones Terrenas}

Provee operaciones para registrar estaciones terrenas, consultarlas y asignar el satélite que deben seguir en un momento dado.

\begin{itemize}
    \item \texttt{GET /api/ground-stations}: Obtener todas las estaciones terrenas registradas.
    \item \texttt{GET /api/ground-stations/\{id\}}: Obtener una estación terrena específica por su identificador.
    \item \texttt{POST /api/ground-stations}: Crear una nueva estación terrena.
    \item \texttt{PUT /api/ground-stations/\{id\}/satellite}: Actualizar el satélite que la estación debe seguir.
\end{itemize}

\paragraph{Telemetría}

La API permite consultar la telemetría decodificada con paginación y administrar el decodificador asociado a cada satélite para ajustar el procesamiento de los paquetes recibidos.

\begin{itemize}
    \item \texttt{GET /api/satellite/\{id\}/telemetry}: Obtener los paquetes de telemetría decodificados de un satélite (parámetros \texttt{pageSize} y \texttt{pageNumber}).
    \item \texttt{GET /api/satellite/\{id\}/telemetry/decoder}: Consultar la configuración del decodificador de telemetría del satélite.
    \item \texttt{PUT /api/satellite/\{id\}/telemetry/decoder}: Actualizar la configuración del decodificador del satélite.
\end{itemize}

\paragraph{Seguimiento}

Estos recursos calculan oportunidades de observación futuras, tanto desde la perspectiva de un satélite como de una estación terrena, permitiendo filtrar por satélites de interés.

\begin{itemize}
    \item \texttt{GET /api/satellites/\{id\}/passes}: Obtener los próximos pases del satélite sobre las estaciones terrenas disponibles.
    \item \texttt{GET /api/ground-stations/\{id\}/passes}: Obtener los próximos satélites que la estación podrá observar.
    \item \texttt{POST /api/ground-stations/\{id\}/passes}: Calcular los próximos satélites a observar para la estación considerando la lista \texttt{sat\_ids}.
\end{itemize}

\paragraph{Jobs}

Los endpoints de trabajos encapsulan la creación y seguimiento de tareas de rastreo, incluyendo la programación de comandos que se ejecutarán sobre la estación terrena.

\begin{itemize}
    \item \texttt{POST /api/jobs}: Crear un nuevo trabajo indicando \texttt{gs\_id}, \texttt{sat\_id} y la lista de comandos.
    \item \texttt{GET /api/jobs}: Listar todos los trabajos registrados.
    \item \texttt{GET /api/jobs/\{id\}}: Obtener la información de un trabajo específico.
    \item \texttt{GET /jobs/\{id\}/status}: Consultar el estado actual del trabajo.
\end{itemize}


\subsection{Message Broker}

Para la comunicación entre las estaciones terrenas y el servidor, se decidió utilizar un \textit{Message Oriented Middleware} (MOM) para manejar el ruteo y delivery de mensajes. En concreto, se optó por utilizar \textbf{MQTT}, que es es estándar para aplicaciones de IoT y circuitos integrados, por lo que resulta adecuado para cualquier hardware que eventualmente utilicen las estaciones terrenas.

El protocolo MQTT soporta una gran variedad de configuraciones para distintos casos de uso. Una de estas configuraciones es la \textbf{calidad de servicio} (QoS), que controla qué garantías de entrega tiene cada mensaje. Como nuestra aplicación tiene poco \textit{throughput} de mensajes y la entrega de varios es crítica, se decidió usar el nivel de QoS más alto, que es QoS 2 (\textit{exactly once}). Con este nivel de QoS el protocolo MQTT garantiza que cada mensaje será entregado una única vez a un receptor. Como en nuestro sistema todos los mensajes son punto a punto, QoS 2 garantiza que todos los mensajes van a llegar al receptor indicado sin duplicación, asegurando que ningún job quede sin asignar y que ninguna actualización del satélite se pierda.

El protocolo MQTT también permite configurar la autenticaión de los participantes por medio de TLS. Esta opción permite añadir una capa de seguridad adicional para garantizar la autenticidad de las estaciones terrenas y mejorar la robustez del sistema en general.

\subsection{Base de Datos}

El servidor se encuentra integrado con una base de datos relacional para llevar registro de las entidades relevantes (estaciones terrenas, satélites, jobs, etc.) así como de la información recibida de las distintas misiones. El esquema actual de la base de datos es el siguiente:

\begin{figure}
    \centering
    \includegraphics[width=1.0\linewidth]{images/bdd_schema.png}
    \caption{Esquema de la base de datos de Rustar}
\end{figure}

El esquema modela los elementos principales del sistema. La tabla \texttt{ground\_stations} registra las estaciones terrenas con su identificador y posición geográfica (latitud, longitud y altitud). La tabla \texttt{satellites} conserva los satélites disponibles junto con su nombre, el TLE vigente y las frecuencias de comunicación. Ambas tablas se enlazan con las distintas misiones programadas en la tabla \texttt{jobs}, que representa una asignación de seguimiento entre una estación y un satélite dentro de un intervalo temporal definido.

Cada \texttt{job} puede incluir múltiples comandos a ejecutar sobre la estación, almacenados en \texttt{job\_commands}. La ejecución de una misión se monitorea mediante \texttt{jobs\_status\_updates}, donde se conservan los cambios de estado más recientes asociados al identificador del job. Finalmente, la tabla \texttt{telemetry} archiva los paquetes de telemetría recibidos, manteniendo la referencia al satélite y a la estación que capturó el dato.

Las claves foráneas garantizan la integridad referencial entre las entidades: los jobs sólo pueden asociarse a satélites y estaciones existentes, los comandos y actualizaciones de estado dependen de un job válido, y los registros de telemetría se vinculan con las mismas entidades que participaron en la captura. Este diseño permite consultar el historial de operaciones y telemetría de forma consistente y auditable.
