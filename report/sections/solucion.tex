\section{Solución implementada}

\subsection{Estación Terrena}

\subsubsection{Antenna Controller}

El módulo Antenna Controller constituye la interfaz de bajo nivel entre el sistema de software y el hardware de orientación de la antena. Su función principal es la traducción de parámetros de apuntado calculados por el módulo de Tracking en comandos compatibles con el protocolo del controlador físico de la antena.

\paragraph{Funcionalidad}

El módulo recibe como entrada los siguientes parámetros de apuntado:
\begin{itemize}
    \item Ángulos de azimut y elevación (en grados decimales)
    \item Identificación del satélite objetivo
    \item Canal o frecuencia de downlink
\end{itemize}

Estos datos son procesados y formateados según el protocolo textual esperado por el controlador de antena. La estructura del comando generado responde al siguiente formato:

\begin{verbatim}
SN=[nombre_satélite],AZ=[azimut],EL=[elevación],DN=[downlink_number]
\end{verbatim}

donde \texttt{AZ} y \texttt{EL} representan los ángulos de orientación expresados en grados con precisión decimal, \texttt{SN} identifica el satélite objetivo y \texttt{DN} especifica el número de canal de downlink.

\paragraph{Implementación}

La comunicación con el hardware se establece mediante puerto serie, utilizando el crate \texttt{serialport} del ecosistema Rust. Esta biblioteca proporciona una API para:
\begin{itemize}
    \item Apertura y configuración de dispositivos serie
    \item Establecimiento de parámetros de comunicación (baudrate, bits de datos, paridad, bits de parada)
    \item Operaciones de lectura y escritura sobre el puerto
\end{itemize}
\subsection{Servidor}

Como parte de la solución, se implementó un servidor con una REST API que funciona como interfaz entre el usuario y las estaciones terrenas. Entre las funciones del servidor se encuentran:

\begin{itemize}
    \item Control remoto de la estación terrena mediante la asignación de tareas de seguimiento.
    \item Recolección de datos de telemetría provenientes del satélite.
    \item Llevar registro de las distintas estaciones terrenas y trabajos asignados.
\end{itemize}

\subsubsection{Comunicación con Estaciones Terrenas}

El servidor tiene un hilo de ejecución dedicado a manejar la comunicación con las estaciones terrenas por medio del broker MQTT. Por medio del broker el servidor puede enviarle jobs a las estaciones terrenas, además de recibir información de control y telemetría de las mismas.

El hilo se encuentra continuamente escuchando eventos MQTT del broker. Cuando recibe un evento de mensaje publicado, extrae el contenido del mensaje y el tópico al que fue publicado. En el nombre del tópico se encuentra la estación terrena del cual provino el mensaje. Con esta información, el mensaje es almacenado en una base de datos para su uso posterior. Cuando el servidor necesita acceder a los mensajes, lo hace directamente por medio de la base de datos.

Para el envío de mensajes, el hilo principal del servidor tiene una copia del cliente que puede usar para enviar mensajes a una estación terrena en particular, especificada por medio del tópico. La librería que utilizamos requiere hacer \textit{polling} constante del cliente para enviar mensajes, el hilo recibidor se encarga de eso.

El broker de MQTT está configurado para garantizar la entrega de mensajes por medio de la \textit{calidad de servicio} (QoS). Utilizamos una calidad de servicio de \textit{exactamente una vez} para asegurarnos de que cada mensaje sea entregado sin duplicación a su destinatario. El broker también puede ser configurado para manejar la autenticación de las estaciones terrenas y el servidor por TLS.

\subsubsection{REST API}

La API expone los servicios que permiten operar el sistema desde aplicaciones externas. Cada conjunto de endpoints agrupa responsabilidades específicas para administrar satélites, estaciones, telemetría, planificación de pases y trabajos de seguimiento.

\paragraph{Gestión de Satélites}

Estos endpoints permiten mantener el catálogo de satélites y sus parámetros orbitales, así como consultar los comandos disponibles para operar cada plataforma.

\begin{itemize}
    \item \texttt{GET /api/satellites}: Obtener todos los satélites con sus datos TLE.
    \item \texttt{GET /api/satellites/\{id\}}: Obtener un satélite específico a partir de su identificador.
    \item \texttt{PUT /api/satellites/\{id\}}: Actualizar el TLE registrado para un satélite.
    \item \texttt{GET /api/satellite/\{id\}/commands}: Consultar los comandos disponibles para el satélite indicado.
\end{itemize}

\paragraph{Gestión de Estaciones Terrenas}

Provee operaciones para registrar estaciones terrenas, consultarlas y asignar el satélite que deben seguir en un momento dado.

\begin{itemize}
    \item \texttt{GET /api/ground-stations}: Obtener todas las estaciones terrenas registradas.
    \item \texttt{GET /api/ground-stations/\{id\}}: Obtener una estación terrena específica por su identificador.
    \item \texttt{POST /api/ground-stations}: Crear una nueva estación terrena.
    \item \texttt{PUT /api/ground-stations/\{id\}/satellite}: Actualizar el satélite que la estación debe seguir.
\end{itemize}

\paragraph{Telemetría}

La API permite consultar la telemetría decodificada con paginación y administrar el decodificador asociado a cada satélite para ajustar el procesamiento de los paquetes recibidos.

\begin{itemize}
    \item \texttt{GET /api/satellite/\{id\}/telemetry}: Obtener los paquetes de telemetría decodificados de un satélite (parámetros \texttt{pageSize} y \texttt{pageNumber}).
    \item \texttt{GET /api/satellite/\{id\}/telemetry/decoder}: Consultar la configuración del decodificador de telemetría del satélite.
    \item \texttt{PUT /api/satellite/\{id\}/telemetry/decoder}: Actualizar la configuración del decodificador del satélite.
\end{itemize}

\paragraph{Seguimiento}

Estos recursos calculan oportunidades de observación futuras, tanto desde la perspectiva de un satélite como de una estación terrena, permitiendo filtrar por satélites de interés.

\begin{itemize}
    \item \texttt{GET /api/satellites/\{id\}/passes}: Obtener los próximos pases del satélite sobre las estaciones terrenas disponibles.
    \item \texttt{GET /api/ground-stations/\{id\}/passes}: Obtener los próximos satélites que la estación podrá observar.
    \item \texttt{POST /api/ground-stations/\{id\}/passes}: Calcular los próximos satélites a observar para la estación considerando la lista \texttt{sat\_ids}.
\end{itemize}

\paragraph{Jobs}

Los endpoints de trabajos encapsulan la creación y seguimiento de tareas de rastreo, incluyendo la programación de comandos que se ejecutarán sobre la estación terrena.

\begin{itemize}
    \item \texttt{POST /api/jobs}: Crear un nuevo trabajo indicando \texttt{gs\_id}, \texttt{sat\_id} y la lista de comandos.
    \item \texttt{GET /api/jobs}: Listar todos los trabajos registrados.
    \item \texttt{GET /api/jobs/\{id\}}: Obtener la información de un trabajo específico.
    \item \texttt{GET /jobs/\{id\}/status}: Consultar el estado actual del trabajo.
\end{itemize}


\subsection{Message Broker}

Para la comunicación entre las estaciones terrenas y el servidor, se decidió utilizar un \textit{Message Oriented Middleware} (MOM) para manejar el ruteo y delivery de mensajes. En concreto, se optó por utilizar \textbf{MQTT}, que es es estándar para aplicaciones de IoT y circuitos integrados, por lo que resulta adecuado para cualquier hardware que eventualmente utilicen las estaciones terrenas.

El protocolo MQTT soporta una gran variedad de configuraciones para distintos casos de uso. Una de estas configuraciones es la \textbf{calidad de servicio} (QoS), que controla qué garantías de entrega tiene cada mensaje. Como nuestra aplicación tiene poco \textit{throughput} de mensajes y la entrega de varios es crítica, se decidió usar el nivel de QoS más alto, que es QoS 2 (\textit{exactly once}). Con este nivel de QoS el protocolo MQTT garantiza que cada mensaje será entregado una única vez a un receptor. Como en nuestro sistema todos los mensajes son punto a punto, QoS 2 garantiza que todos los mensajes van a llegar al receptor indicado sin duplicación, asegurando que ningún job quede sin asignar y que ninguna actualización del satélite se pierda.

El protocolo MQTT también permite configurar la autenticaión de los participantes por medio de TLS. Esta opción permite añadir una capa de seguridad adicional para garantizar la autenticidad de las estaciones terrenas y mejorar la robustez del sistema en general.

\subsection{Base de Datos}

El servidor se encuentra integrado con una base de datos relacional para llevar registro de las entidades relevantes (estaciones terrenas, satélites, jobs, etc.) así como de la información recibida de las distintas misiones. El esquema actual de la base de datos es el siguiente:

\begin{figure}
    \centering
    \includegraphics[width=1.0\linewidth]{images/bdd_schema.png}
    \caption{Esquema de la base de datos de Rustar}
\end{figure}

El esquema modela los elementos principales del sistema. La tabla \texttt{ground\_stations} registra las estaciones terrenas con su identificador y posición geográfica (latitud, longitud y altitud). La tabla \texttt{satellites} conserva los satélites disponibles junto con su nombre, el TLE vigente y las frecuencias de comunicación. Ambas tablas se enlazan con las distintas misiones programadas en la tabla \texttt{jobs}, que representa una asignación de seguimiento entre una estación y un satélite dentro de un intervalo temporal definido.

Cada \texttt{job} puede incluir múltiples comandos a ejecutar sobre la estación, almacenados en \texttt{job\_commands}. La ejecución de una misión se monitorea mediante \texttt{jobs\_status\_updates}, donde se conservan los cambios de estado más recientes asociados al identificador del job. Finalmente, la tabla \texttt{telemetry} archiva los paquetes de telemetría recibidos, manteniendo la referencia al satélite y a la estación que capturó el dato.

Las claves foráneas garantizan la integridad referencial entre las entidades: los jobs sólo pueden asociarse a satélites y estaciones existentes, los comandos y actualizaciones de estado dependen de un job válido, y los registros de telemetría se vinculan con las mismas entidades que participaron en la captura. Este diseño permite consultar el historial de operaciones y telemetría de forma consistente y auditable.

\subsection{Interfaz de Usuario}

La interfaz de usuario del sistema proporciona una plataforma web moderna e intuitiva para la gestión y monitoreo de satélites y estaciones terrenas. Desarrollada con tecnologías web modernas (\textit{Next.js}, \textit{React} y \textit{TypeScript}), la interfaz ofrece una experiencia fluida, responsiva y accesible desde cualquier navegador web, adaptándose automáticamente a diferentes dispositivos desde teléfonos móviles hasta monitores de escritorio.

La aplicación se estructura en dos módulos principales: Satélites y Estaciones Terrenas. El usuario puede alternar entre estos módulos mediante una barra de navegación superior. La interfaz se comunica con la API del servidor para obtener y enviar datos en tiempo real, garantizando que la información presentada esté siempre actualizada.

\subsubsection{Módulo de Satélites}

Al acceder al módulo de satélites, el usuario visualiza todos los satélites disponibles en el sistema. Esta vista permite identificar rápidamente cada satélite bajo control y acceder a sus funcionalidades específicas de monitoreo y control.

\paragraph{Seguimiento Orbital}

El usuario puede visualizar en tiempo real la posición de cualquier satélite sobre un mapa interactivo de la Tierra. El sistema calcula y actualiza continuamente la trayectoria orbital, mostrando la ubicación actual del satélite segundo a segundo. Junto al mapa, se presenta información precisa en tiempo real: coordenadas geográficas exactas (latitud y longitud) y altitud sobre el nivel del mar en metros.

Una funcionalidad importante es la capacidad de actualizar los parámetros orbitales del satélite. Cuando se obtienen datos TLE más recientes y precisos, el usuario puede ingresarlos directamente en el sistema, mejorando inmediatamente la exactitud del seguimiento orbital.

\paragraph{Monitoreo de Telemetría}

Esta sección permite supervisar en tiempo real el estado de salud y funcionamiento del satélite. Los datos de telemetría se obtienen continuamente desde la API y se presentan mediante gráficos que muestran la evolución temporal de parámetros críticos como temperatura interna y voltaje del sistema eléctrico.

La interfaz presenta tres perspectivas complementarias: gráficos visuales para identificar tendencias y anomalías, un visor de datos en formato crudo que muestra el flujo de paquetes tal como llegan del satélite, y una tabla estructurada que lista los paquetes recibidos con su hora exacta de recepción, identificador, tamaño y estado.

\paragraph{Centro de Comandos}

El centro de comandos proporciona la capacidad de enviar instrucciones al satélite de forma controlada y segura. El sistema presenta una lista de comandos predefinidos. Los comandos seleccionados se transmiten a través de la API hacia el sistema de control del satélite.

Un historial completo registra todos los comandos enviados, mostrando para cada uno su estado actual: pendiente (en espera de transmisión), enviado (transmitido exitosamente) o fallido (error en la transmisión). Este registro permite rastrear todas las operaciones realizadas y verificar que los comandos fueron enviados al satélite según lo previsto.

\paragraph{Predicción de Pases}

Esta funcionalidad permite visualizar los próximos pases del satélite seleccionado sobre las estaciones terrenas disponibles. El sistema calcula y presenta las ventanas de visibilidad para las próximas 48 horas, mostrando cuándo y desde qué estación terrena será posible establecer comunicación con el satélite.

La información se presenta en formato de timeline visual y tabla detallada, indicando para cada pase la estación terrena correspondiente, hora de inicio y fin de visibilidad, duración del pase y elevación máxima alcanzada.

\subsubsection{Módulo de Estaciones Terrenas}

El módulo de estaciones terrenas presenta todas las instalaciones disponibles en el sistema, permitiendo acceder rápidamente a la información y funcionalidades de cada estación.

\paragraph{Ubicación y Detalles}

Al seleccionar una estación terrena, el usuario accede a un mapa centrado en su ubicación exacta. La vista proporciona información precisa de posicionamiento: latitud y longitud con cuatro decimales de precisión, y altitud sobre el nivel del mar expresada en metros.

\paragraph{Planificación de Pases}

Una de las funcionalidades más importantes para las operaciones diarias es la visualización de pases satelitales. El sistema calcula automáticamente todos los momentos en que diferentes satélites serán visibles desde la estación terrena seleccionada, proyectando las próximas 48 horas.

El usuario puede consultar esta información de dos maneras complementarias. Un \textit{timeline} visual horizontal representa gráficamente la distribución temporal de todos los pases, permitiendo identificar períodos de máxima actividad, detectar solapamientos entre pases de diferentes satélites y localizar ventanas de oportunidad libres para programar comunicaciones específicas.

Complementariamente, una tabla detallada lista cada pase con información específica: satélite correspondiente, hora exacta de AOS (inicio de visibilidad), hora de LOS (fin de visibilidad), duración total del pase en minutos, y elevación máxima que alcanzará el satélite sobre el horizonte. Esta elevación máxima es un dato crítico, ya que pases con mayor elevación generalmente ofrecen mejor calidad de comunicación.

Ambas visualizaciones están sincronizadas: cuando el usuario desplaza el cursor sobre un pase en la tabla, el mismo pase se resalta visualmente en el timeline, y viceversa, facilitando la correlación entre la información tabular precisa y la representación temporal gráfica.
