El desarrollo de sistemas para el segmento terreno de misiones satelitales ha evolucionado significativamente en las últimas décadas, especialmente con el auge de los \textit{CubeSats} y la democratización del acceso al espacio. En este contexto, existen diversas herramientas y plataformas utilizadas tanto en ambientes académicos como en la industria para tareas de Telemetría, Seguimiento y Control (\textit{TT\&C}). A continuación se describen las principales alternativas existentes y sus características relevantes.

\subsection{Herramientas de procesamiento de señales}

\textbf{GNU Radio} es una de las herramientas de \textit{SDR} más utilizadas en el ámbito de las comunicaciones satelitales. Se trata de un \textit{framework} de código abierto altamente flexible y extensible que permite implementar sistemas de procesamiento de señales mediante bloques modulares. Su principal fortaleza radica en su capacidad para procesar señales en tiempo real y su amplia biblioteca de bloques predefinidos. Sin embargo, presenta una curva de aprendizaje pronunciada y no cuenta con una interfaz orientada específicamente a operaciones \textit{TT\&C}, requiriendo que los usuarios integren manualmente componentes adicionales para tareas de seguimiento orbital y gestión de telemetría.

\subsection{Software de seguimiento orbital}

Para el seguimiento y predicción de órbitas satelitales, existen herramientas especializadas como \textbf{Orbitron} y \textbf{GPredict}. Estas aplicaciones son ampliamente utilizadas para el cálculo de efemérides y la visualización de trayectorias orbitales, proporcionando información crucial sobre las ventanas de visibilidad de los satélites. \textbf{GPredict}, en particular, es una solución de código abierto que ofrece capacidades de predicción de órbitas y control básico de rotores de antena mediante el protocolo \textit{Hamlib}. Sin embargo, estas herramientas no se integran directamente con sistemas de procesamiento de señales ni con plataformas de gestión de telemetría, requiriendo operación manual y coordinación entre múltiples aplicaciones.

\subsection{Plataformas integradas}

\textbf{SatNOGS} (Satellite Networked Open Ground Station) representa uno de los esfuerzos más ambiciosos en el ámbito de estaciones terrenas de código abierto. Desarrollado por la \textit{Libre Space Foundation}, SatNOGS es una red global de estaciones terrenas automatizadas que permite la recopilación colaborativa de datos satelitales. El proyecto incluye diseños de hardware de bajo costo, software de control de estación y una base de datos centralizada de información satelital. Si bien ofrece una infraestructura avanzada y una comunidad activa, su enfoque principal está orientado hacia redes colaborativas y la recopilación pasiva de datos de múltiples satélites. Esta característica limita su aplicabilidad para operaciones dedicadas que requieren control activo intensivo de \textit{CubeSats} específicos en entornos académicos o de desarrollo de misiones.

\subsection{Soluciones comerciales}

En el ámbito comercial, existen escasas plataformas propietarias como por ejemplo \textbf{AWS Ground Station} que ofrecen capacidades completas de \textit{TT\&C}. Estas soluciones proporcionan interfaces integradas y soporte profesional, pero requieren licencias costosas y no permiten la personalización profunda necesaria para entornos de investigación y desarrollo. Además, su naturaleza cerrada las hace inadecuadas para proyectos académicos que buscan fomentar el aprendizaje y la experimentación.