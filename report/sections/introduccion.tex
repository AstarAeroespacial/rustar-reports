El presente documento describe el desarrollo de RUSTAR, un \textit{framework} de código abierto para el segmento terreno de misiones satelitales, con especial énfasis en tareas de Telemetría, Seguimiento y Control (\textit{TT\&C}). Este trabajo se enmarcó en el proyecto Astar, una iniciativa del Laboratorio Abierto (LABi) de la Facultad de Ingeniería de la Universidad de Buenos Aires orientada a la formación práctica y la investigación en el ámbito aeroespacial, particularmente en el desarrollo de satélites tipo \textit{CubeSat}.

Antes del inicio del proyecto, la estación terrena física de Astar dependía de múltiples herramientas de software independientes para su operación. Cada una de estas herramientas, como \textit{GNU Radio} para el procesamiento de señales u \textit{Orbitron} para el seguimiento orbital, debía ser operada de forma manual y coordinada, lo que fragmentaba el flujo de trabajo, incrementaba la complejidad operativa y dificultaba la formación de nuevos usuarios. Esta situación evidenció la necesidad de contar con una plataforma integrada que unificara las funcionalidades esenciales del segmento terreno.

Ante este contexto, el objetivo del presente trabajo profesional fue desarrollar RUSTAR como un \textit{framework} integral que unificara en una sola plataforma todas las funcionalidades necesarias para operar estaciones terrenas. A diferencia de una solución específica para un único caso de uso, RUSTAR fue concebido como un sistema modular y extensible que permite su adaptación a diferentes configuraciones de hardware y distintos tipos de misiones satelitales. Esta solución busca centralizar y simplificar las tareas de seguimiento, recepción y transmisión de datos, facilitando el control de las misiones satelitales y mejorando significativamente la experiencia de los operadores.

El \textit{framework} está basado completamente en tecnologías de código abierto y fue desarrollado siguiendo principios de diseño modular, lo que permite su reutilización y extensión por parte de la comunidad científica y académica. Además, incorpora una interfaz gráfica de usuario diseñada para brindar una experiencia clara e intuitiva a estudiantes, docentes, investigadores y operadores de misiones satelitales.

Este documento presenta el trabajo realizado, describiendo tanto los fundamentos teóricos como las decisiones de diseño e implementación que guiaron el desarrollo de RUSTAR, así como los resultados obtenidos y las lecciones aprendidas durante el proceso.