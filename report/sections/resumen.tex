El presente trabajo profesional tuvo como objetivo el desarrollo de RUSTAR, un \textit{framework} de código abierto para el segmento terreno de misiones satelitales. El desarrollo se realizó en el contexto del proyecto Astar de la Facultad de Ingeniería de la Universidad de Buenos Aires, que trabaja en el desarrollo de satélites de tipo \gls{cubesat}, aunque el sistema se encuentra diseñado para ser fácilmente extensible a cualquier tipo de satélite. RUSTAR permite la operación óptima, desde el punto de vista del software, de estaciones terrenas dedicadas a tareas de \gls{telemetriaseguimientoycontrol} (\gls{ttc}), facilitando la comunicación con satélites mediante dispositivos de \gls{radiodefinidaporsoftware} (\gls{SDR}) e implementando las modulaciones y demodulaciones necesarias para el procesamiento de señales. El sistema incorpora funciones de seguimiento satelital en tiempo real, permitiendo que el sistema de rotación del brazo robótico que controla la antena apunte correctamente al satélite durante su paso orbital.

Para garantizar el intercambio de datos de forma segura, eficiente y confiable, se implementó un protocolo de comunicaciones adecuado. El sistema es accesible de forma remota, dada la posibilidad de que los operadores y la estación se encuentren en ubicaciones distintas. Se desarrolló una interfaz gráfica de usuario para facilitar la visualización del estado de la misión y el envío de comandos. El \textit{framework} fue diseñado con un enfoque modular y extensible, basado completamente en herramientas de código abierto, permitiendo su adaptación a distintas configuraciones y necesidades de diferentes misiones satelitales.

La finalidad de este trabajo fue ofrecer una plataforma integral y reutilizable que optimice el accionar de los operadores y facilite la interacción con satélites en diferentes misiones. Además, al estar vinculado con el entorno académico y ser de código abierto, el \textit{framework} contribuye al fortalecimiento de la formación práctica en el área aeroespacial, sirviendo como herramienta de aprendizaje y experimentación para estudiantes e investigadores de la universidad y la comunidad científica en general.