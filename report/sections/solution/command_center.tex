\subsubsection{Servidor}

Como parte de la solución, se implementó una aplicación que funciona como el \textit{backend} del \gls{centrocontrol}. Es un servidor que funciona como intermediario entre los operadores de la misión y las estaciones terrenas. Entre las funciones del servidor se encuentran:

\begin{itemize}
    \item Control remoto de la estación terrena mediante la asignación de tareas de seguimiento.
    \item Recolección de datos de telemetría provenientes del satélite.
    \item Llevar registro de las distintas estaciones terrenas y trabajos asignados.
\end{itemize}

El servidor maneja la comunicación con las estaciones terrenas mediante \gls{mqtt}. Se encuentra continuamente escuchando los tópicos de interés, para extraer y obtener los mensajes, que pueden ser telemetría de algún satélite, actualizaciones de estado de las tareas de observación asignadas, o información de estado de las estaciones terrenas. Una vez obtenida la información, se almacena en la base de datos para ser utilizada posteriormente. Cuando el servidor necesita acceder a los datos recogidos, lo hace directamente por medio de la base de datos.

Además, mediante \gls{mqtt} se envían las tareas de observación a las estaciones terrenas correspondientes.

El servidor se encuentra integrado con una base de datos relacional para llevar registro de las entidades relevantes (estaciones terrenas, satélites, jobs, etc.) así como de la información recibida de las distintas misiones. Los detalles de la base de datos se encuentran en Anexo~\ref{sec:database}.

El servidor expone una API HTTP que permite a servicios externos operar el sistema. Los "servicios" que provee la API se agrupan de la siguiente forma:

\begin{itemize}
    \item Gestión de satélites: para gestionar el catálogo de satélites de la misión, junto a sus atributos, comandos disponibles y otros.
    \item Gestión de estaciones terrenas: para gestionar las estaciones terrenas que proveen servicios a la misión, sus atributos y su estado.
    \item Telemetría: para consultar la telemetría histórica y en tiempo real de un satélite.
    \item Seguimiento: para conocer las oportunidades de observación futuras, tanto desde el punto de vista de un satélite como de una estación terrena.
    \item Tareas de observación: para crear y monitorear tareas de observación.
\end{itemize}

Se detallan los endpoints disponibles al momento de la publicación de este documento en Anexo~\ref{sec:api}.

\subsubsection{Interfaz gráfica de usuario}

La interfaz gráfica de usuario permite a los operadores de la misión interactuar con el sistema de forma intuitiva y centralizada. A través de esta aplicación, se accede a las funciones expuestas por el servidor, brindando una visión unificada del estado del segmento terreno y del conjunto de satélites operados.

La interfaz ofrece herramientas para visualizar el estado en tiempo real de las estaciones terrenas, monitorear su disponibilidad y funcionamiento, y consultar la telemetría recibida de cada satélite. También permite explorar las oportunidades de observación futuras mediante representaciones gráficas de las órbitas y las ventanas de visibilidad, facilitando la planificación de tareas de seguimiento.

Además, la interfaz posibilita la creación, asignación y supervisión de tareas de observación. Los operadores pueden seleccionar un satélite, elegir la estación terrena deseada y configurar los parámetros de la operación, delegando luego la ejecución en el centro de control y en la estación correspondiente. La aplicación muestra el progreso de cada tarea y las actualizaciones de estado enviadas por las estaciones a través de la cola de mensajería.

En conjunto, la interfaz gráfica constituye el punto de acceso principal para la operación de la misión, integrando los datos procesados por el servidor con herramientas visuales que simplifican la supervisión, la planificación y el control de las actividades del segmento terreno. Su diseño facilita la interacción con un sistema complejo, reduciendo la carga cognitiva de los operadores y permitiendo una operación más eficiente y confiable.

En el Anexo~\ref{sec:front} se detallan las tecnologías utilizadas y el estado actual del desarrollo de la interfaz de usuario.