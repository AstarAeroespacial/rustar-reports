Para la comunicación entre las estaciones terrenas y el servidor, se decidió utilizar un \textit{Message Oriented Middleware} (MOM) para manejar el \textit{routing} y \textit{delivery} de mensajes. En concreto, se optó por utilizar \gls{mqtt}, que es es estándar para aplicaciones de IoT y sistemas embebidos, por lo que resulta adecuado para cualquier hardware que eventualmente utilicen las estaciones terrenas.

El protocolo \gls{mqtt} soporta una gran variedad de configuraciones para distintos casos de uso. Una de estas configuraciones es la calidad de servicio (QoS), que controla qué garantías de entrega tiene cada mensaje. Como nuestra aplicación tiene poco \textit{throughput} de mensajes y la entrega de estos es crítica, se decidió usar el nivel de QoS más alto, que es QoS 2 (\textit{exactly once}). Con este nivel de QoS el protocolo \gls{mqtt} garantiza que cada mensaje será entregado una única vez a un receptor. Como en nuestro sistema todos los mensajes son punto a punto, QoS 2 garantiza que todos los mensajes van a llegar al receptor indicado sin duplicación, asegurando que ningún job quede sin asignar y que ninguna actualización del satélite se pierda.

El protocolo \gls{mqtt} también permite configurar la autenticaión de los participantes por medio de TLS. Esta opción permite añadir una capa de seguridad adicional para garantizar la autenticidad de las estaciones terrenas y mejorar la robustez del sistema en general.
