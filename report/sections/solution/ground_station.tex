Las funciones que lleva a cabo para cumplir con su propósito son:
\begin{itemize}
    \item Comunicarse con el dispositivo \gls{SDR}, encargado de la transmisión y recepción de señales de radiofrecuencia.
    \item Realizar la demodulación y modulación necesarios para convertir las señales en frames de bits y viceversa.
    \item Implementar el protocolo de enlace de datos requerido para el intercambio de información.
    \item Controlar la orientación de la antena para seguir la trayectoria del satélite.
\end{itemize}

Estas funciones se ejecutan durante una ventana de observación de un satélite con el cual se quiere establecer contacto. Para programar un contacto u observación, el mecanismo provisto por la estación terrena es mediante el envío de una tarea de observación.

\subsubsection{Tareas de observación}

Las tareas de observación se envían a las estaciones terrenas a través de la cola de mensajería.

Cada tarea de observación contiene la información necesaria para llevar a cabo la operación:
\begin{itemize}
    \item Identificador del satélite, necesario para correlacionar la telemetría e información obtenida.
    \item Los parámetros orbitales para calcular la posición y velocidad relativa del satélite.
    \item Las frecuencias de transmisión y recepción para configurar el radioenlace.
    \item La ventana temporal de observación, con la adquisición y pérdida de señal.
    \item Opcionalmente, información para transmitir al satélite en caso de ser necesario.
\end{itemize}

Una vez recibida la tarea de observación, la estación programará su ejecución para que comience automáticamente al iniciar la ventana de contacto. Además, reporta el estado de la tarea mediante la cola de mensajería.

Durante la ejecución de la observación, se obtiene la telemetría transmitida por el satélite y se envía a la cola de mensajería. Si hubiere, se envían al satélite los comandos contenidos en la tarea de observación, asegurando su correcta recepción.

\subsubsection{Radio definida por software}

La aplicación de la estación terrena se comunica directamente con el dispositivo \gls{SDR} para configurar los parámetros operativos del enlace, principalmente las frecuencias de transmisión y recepción, el ancho de banda y la tasa de muestreo. Durante la recepción, el \gls{SDR} entrega un flujo continuo de muestras crudas que representan la señal capturada en el dominio digital, las cuales son procesadas por las etapas de demodulación. De forma análoga, en transmisión, la aplicación genera las muestras correspondientes a la señal modulada y las envía al \gls{SDR} para su conversión y emisión en radiofrecuencia.

La comunicación con el \gls{SDR} se realiza a través de una capa de abstracción que define una interfaz común para los distintos dispositivos. Esto permite utilizar diferentes dispositivos \gls{SDR} o fuentes de muestras sin modificar la lógica principal del sistema. Gracias a esta separación, la estación puede adaptarse a diversos entornos de hardware y esquemas de modulación, manteniendo una arquitectura flexible y fácilmente extensible.

La capa de abstracción para interactuar con el dispositivo \gls{SDR} se implementa utilizando \gls{soapysdr}. A través de su interfaz, se puede inicializar y configurar una gran variedad de dispositivos \gls{SDR}, para luego enviar y recibir muestras de la señal digitalizada.

\subsubsection{Modulación y demodulación}

Para el intercambio de información entre las estaciones terrenas y los satélites, la estación terrena implementa la modulación y demodulación digital responsable de transformar los datos en una forma de onda adecuada para la transmisión y de recuperar la información recibida a partir de las muestras capturadas. En transmisión, se generan secuencias de muestras digitales según el esquema de modulación definido para la misión, mientras que en recepción se realiza el proceso inverso, decodificando las muestras provenientes del \gls{SDR} para reconstruir el flujo original de bits. Estas operaciones constituyen la base de la capa física del enlace y permiten establecer el intercambio de información entre ambos extremos de la comunicación.

El procesamiento digital de señales involucrado en la modulación y demodulación, incluyendo operaciones como correlación, filtrado, sincronización y detección, se implementa utilizando \gls{gnuradio}. La aplicación de la estación terrena utiliza una interfaz que permite instanciar, configurar y ejecutar los flujos de \gls{gnuradio}, los cuales realizan el procesamiento correspondiente durante la recepción o la transmisión. Este enfoque integra las etapas de procesamiento digital dentro del ciclo de ejecución de cada observación, manteniendo una separación clara entre la lógica de control y el procesamiento de señales, y permitiendo reutilizar o adaptar fácilmente los flujos de \gls{gnuradio} para diferentes misiones o esquemas de modulación.

La integración con \gls{gnuradio} se realiza ejecutando los flowgraphs definidos en Python como procesos externos. La aplicación inicializa y controla la ejecución de estos procesos, permitiendo elegir de forma dinámica el flujo a utilizar en función del esquema de modulación implementado por la misión. El intercambio de muestras y bits entre la aplicación y el flowgraph se realiza mediante \gls{zeromq}.

\subsubsection{Protocolo de enlace}

Una vez que las etapas de modulación y demodulación convierten las señales entre el dominio analógico y digital, es necesario implementar un protocolo de capa de enlace de datos que estructure el flujo de bits en unidades significativas de información. La arquitectura de la aplicación de la estación terrena está diseñada para permitir la integración de distintos protocolos de enlace según los requerimientos de cada misión. Para el presente desarrollo se implementó el protocolo \gls{HDLC} (High-Level Data Link Control), el cual constituye el estándar de facto en comunicaciones satelitales de órbita baja y es ampliamente utilizado en proyectos de CubeSats y satélites de radioaficionados. Una descripción detallada del protocolo \gls{HDLC}, su estructura de frames y mecanismos de transparencia se encuentra en el anexo correspondiente.

La implementación del protocolo se organiza en dos componentes principales: el \textit{framer} y el \textit{deframer}. El \textit{framer} es responsable de estructurar los datos para la transmisión, mientras que el \textit{deframer} realiza el proceso inverso en recepción. Durante la transmisión, los datos son encapsulados en frames que incluyen delimitadores de sincronización y campos de verificación de integridad. En recepción, el flujo de bits es analizado para identificar los límites de cada frame, extraer su contenido y validar su correcta recepción antes de entregar los datos a las capas superiores del sistema.

El sistema está diseñado para operar de manera robusta en condiciones adversas del canal de comunicación, características típicas de los enlaces satelitales. Esto incluye la capacidad de resincronizarse automáticamente ante pérdida de señal, descartar datos corruptos o inválidos, y procesar el flujo de información de manera continua sin pérdida de frames válidos.


\subsubsection{Seguimiento}

La función de seguimiento es fundamental para el funcionamiento de la estación terrena, ya que permite seguir la trayectoria de los satélites durante las ventanas de observación. Su propósito principal es calcular, en tiempo real, los parámetros necesarios para mantener el enlace con el satélite a medida que este se desplaza en su órbita.

Durante la ejecución de una tarea de observación, se utilizan los parámetros orbitales del satélite para determinar continuamente:

\begin{itemize}
    \item Los ángulos de acimut y elevación necesarios para orientar la antena y mantenerla apuntando al satélite.
    \item La corrección de frecuencia por \gls{efectodoppler}, tanto para la transmisión (\textit{uplink}) como para la recepción (\textit{downlink}), compensando el desplazamiento de frecuencia causado por el movimiento relativo entre la estación y el satélite.
\end{itemize}

Estos valores se actualizan periódicamente durante toda la ventana de observación. Los ángulos calculados se envían al controlador de antena para ejecutar el seguimiento físico del satélite, mientras que las correcciones de frecuencia se aplican al dispositivo \gls{SDR} para mantener sintonizadas las frecuencias de comunicación a pesar del \gls{efectodoppler}.

La implementación del módulo de seguimiento se describe en detalle en el Anexo~\ref{sec:tracking}.

\subsubsection{Controlador de antena}

El módulo Antenna Controller constituye la interfaz de bajo nivel entre el sistema de software y el hardware de orientación de la antena. Su función principal es la traducción de parámetros de apuntado calculados por el módulo de Tracking en comandos compatibles con el protocolo del controlador físico de la antena.

\paragraph{Funcionalidad}

El módulo recibe como entrada los siguientes parámetros de apuntado:
\begin{itemize}
    \item Ángulos de azimut y elevación (en grados decimales)
    \item Identificación del satélite objetivo
    \item Canal o frecuencia de downlink
\end{itemize}

Estos datos son procesados y formateados según el protocolo textual esperado por el controlador de antena. La estructura del comando generado responde al siguiente formato:

\begin{verbatim}
SN=[nombre_satélite],AZ=[azimut],EL=[elevación],DN=[downlink_number]
\end{verbatim}

donde \texttt{AZ} y \texttt{EL} representan los ángulos de orientación expresados en grados con precisión decimal, \texttt{SN} identifica el satélite objetivo y \texttt{DN} especifica el número de canal de downlink.

\paragraph{Implementación}

La comunicación con el hardware se establece mediante puerto serie, utilizando el crate \texttt{serialport} del ecosistema Rust. Esta biblioteca proporciona una API para:
\begin{itemize}
    \item Apertura y configuración de dispositivos serie
    \item Establecimiento de parámetros de comunicación (baudrate, bits de datos, paridad, bits de parada)
    \item Operaciones de lectura y escritura sobre el puerto
\end{itemize}
