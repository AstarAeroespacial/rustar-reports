\subsection{Glosario}
\textit{Analog-Digital Converter (ADC):} Conversor analógico-digital. Es un dispositivo de hardware que transforma señales analógicas a un formato digital (binario).

\textit{CubeSat:} Satélite pequeño, compuesto de unidades cúbicas de 10 cm de lado y de masa menor a 2 kg, generalmente construidos con componentes estándar de mercado. Definido en el estándar ISO 17770:2017.

% \textit{Digital-Analog Converter (DAC):} Conversor digital-analógico. Es un dispositivo que transforma señales representadas en formato digital (binario) a su equivalente analógico.

\textit{Estación terrena:} Infraestructura terrestre equipada para comunicarse con satélites mediante enlaces de radiofrecuencia. Se encarga de transmitir comandos, recibir datos de telemetría y rastrear la trayectoria orbital de los satélites.

\textit{Field Programmable Gate Array (FPGA):} Circuito integrado con la capacidad de ser reprogramado varias veces de forma sencilla. Pueden ser configurados para realizar distintas funciones lógicas y permiten el desarrollo de hardware y software de forma simultánea.

\textit{In-Phase and Quadrature components (IQ):} Representación de una señal sinusoidal como suma de 2 señales ortogonales ('en fase' y 'cuadratura'). Permite representar una señal analógica utilizando pares de números.

\textit{Software Defined Radio (SDR):} Radio definida por software. Se trata de un sistema compuesto de un receptor de radiofrecuencia, un conversor análogico-digital y un programa de computadora. Este último se encarga de realizar por medio de software diversos tipos de procesamiento sobre la señal que antiguamente se solían hacer por medio de hardware.

\textit{TT\&C (Telemetría, Seguimiento y Control):} conjunto de funciones esenciales en operaciones satelitales que permite monitorear el estado del satélite (telemetría), calcular y predecir su posición (seguimiento), y enviarle comandos desde la estación terrena (control).

\textit{Two Line Element (TLE):}\label{gls:TLE} Formato estándar para la codificación de información sobre objetos en órbita. Utilizando la data contenida en un TLE, se puede calcular la velocidad y posición de un satélite en cualquier instante.

\textit{SGP4 (Simplified General Perturbations 4):}\label{gls:SGP4} propagador orbital estándar utilizado con TLEs para estimar posición y velocidad del satélite.
