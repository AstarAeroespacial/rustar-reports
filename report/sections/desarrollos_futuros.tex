% Desarrollos futuros
\section{Desarrollo Futuro}

En la siguiente sección se detallan algunas posibles features y mejoras que identificamos y que resultarían convenientes añadir en versiones futuras del proyecto

\begin{itemize}
    \item\textbf{Tests del uplink:} antes de la puesta en órbita del satélite, habría que validar el flujo completo de subida con alguna misión ya en órbita, que no pudimos hacer por restricciones de tiempo. El \textit{downlink} ya fue validado con pruebas en el laboratorio, pero de todas formas hacer pruebas completas del \textit{roundtrip} con un satélite real sería valioso para el desarrollo futuro.
    \item\textbf{Integración de comandos:} el sistema actual tiene implementados comandos genéricos como prueba de concepto. Más avanzada la misión, junto con el equipo a cargo de la computadora de a bordo del satélite habría que definir qué comandos van a ser utilizados durante la misión, e implementar los mismos en la interfaz de usuario para su fácil ejecución.
    \item\textbf{Deserialización de datos del satélite:} con ánimos de mantener el proyecto lo suficientemente flexible para varias misiones satelitales, los datos recibidos del satélite se guardan en \textit{crudo} en binario en la base de datos, y que la deserialización se haga en la interfaz de usuario que es un componente más específico y fácil de alterar. Se contempló integrar una interfaz para definir el formato de deserialización de datos de cada satélite, y que esta información sea almacenada en conjunto con el satélite. De este modo, el servidor podría deserializar los datos recibidos del satélite y almacenarlos en un formato más amigable como texto plano o JSON.
    \item\textbf{Integración de nuevos protocolos:} Esta primera versión implementó únicamente asfk1200 como protocolo de modulación y HDLC como protocolo de framing. Se puede añadir soporte para más protocolos de modulación y framing para dar soporte a otras misiones, el sistema ya tiene interfaces estandarizadas para facilitar su integración.
    \item\textbf{Protocolos de encriptación:} Se podrían añadir capas de encriptación adicionales (p.ej. encriptación de extremo a extremo entre el satélite y la interfaz) para mejorar la seguridad del sistema. También se pueden agregar otras medidas de seguridad a otros componentes como el frontend o la REST API.
    \item\textbf{Despliegue definitivo para Astar:} Una vez más avanzado el proyecto Astar, se deberían desplegar definitivamente las aplicaciones. La que requiere principal atención es la estación terrena, ya que tiene que correr en hardware conjunto a la antena en el lugar definitivo donde vaya a ser montada. También se puede discutir la posibilidad de desplegar los otros componentes en premisas propias en vez de en un ambeinte cloud como se tienen acctualmente, para tener control más centralizado de la misión. 
    \item\textbf{Otras tareas relacionadas a Astar:} durante el desarrollo del proyecto, descubrimos varias áreas de la misión Astar que requieren de más trabajo y desarrollo, como la organización del conocimiento y demás recursos digitales del proyecto, el desarrollo de una página web de presentación, y mejoras al código de componentes existentes como la antena. Si bien estas tareas no son parte directa del proyecto, al ser Astar el primer usuario final está en los intereses del proyecto proveer apoyo para fomentar la colaboración continua y obtener feedback de utilidad para las distintas aplicaciones.
\end{itemize}

Se ha discutido con el equipo seguir el desarrollo del proyecto más allá de la entrega del Trabajo Profesional, para seguir brindando apoyo tanto a Astar como a otros proyectos universitarios similares. Consideramos que la propuesta de valor es muy positiva y podría ser de interés para futuros grupos de estudiantes el seguir aportando a la misma.
