% Definiciones del glosario, importar al principio (no imprime nada).

%SDR
\newglossaryentry{softwaredefinedradio}{
        name={\textit{software-defined radio}},
        description={Sistema que utiliza procesamiento digital para funciones de radiofrecuencia, en lugar del tradicional procesamiento de señales en hardware.}
}

\newglossaryentry{SDR}{
        name=SDR,
        description={Ver \gls{softwaredefinedradio}.}
}

\newglossaryentry{radiodefinidaporsoftware}{
        name={radio definida por software},
        description={Ver \gls{softwaredefinedradio}.}
}

\newglossaryentry{gnuradio}{
    name={GNU Radio},
    description={Conjunto de herramientas de desarrollo de código abierto que proporciona bloques de procesamiento de señales para implementar radios definidas por software.}
}

%TLE
\newglossaryentry{twolineelement}{
        name={\textit{two-line element set}},
        description={Formato estandarizado que describe los elementos orbitales de un objeto en órbita, mediante dos líneas de datos.}
}

\newglossaryentry{tle}{
        name=TLE,
        description={Ver \gls{twolineelement}.}
}

% TT&C
\newglossaryentry{ttc}{
        name={TT\&C},
        description={Ver \gls{telemetrytrackingandcontrol}.}
}

\newglossaryentry{telemetrytrackingandcontrol}{
        name={\textit{telemetry, tracking and control}},
        description={Conjunto de funciones que permite monitorear el estado del satélite (telemetría), calcular y predecir su posición (seguimiento) y enviarle comandos (control).}
}

\newglossaryentry{telemetrytrackingandcommand}{
        name={\gls{telemetry, tracking and command}},
        description={Ver \gls{telemetrytrackingandcontrol}}
}

\newglossaryentry{telemetriaseguimientoycontrol}{
        name={telemetría, seguimiento y control},
        description={Ver \gls{telemetrytrackingandcontrol}}
}

%Cubesat
\newglossaryentry{cubesat}{
        name={CubeSat},
        description={Estándar de diseño de nanosatélites, compuesto de unidades cúbicas de 10 cm por lado y masa inferior a 1,33 kg. Definido en el estándar ISO 17770:2017.}
}

%Orbitron
\newglossaryentry{orbitron}{
        name={Orbitron},
        description={Software de seguimiento satelital disponible para Windows que, a partir de un \gls{tle}, calcula posiciones, acimut/elevación y ventanas de visibilidad.}
}

%Gpredict
\newglossaryentry{gpredict}{
        name={Gpredict},
        description={Software libre de seguimiento satelital multiplataforma que permite predecir y visualizar el paso de satélites, calcular parámetros orbitales y controlar rotores de antenas.}
}

%Satnogs
\newglossaryentry{satnogs}{
        name={SatNOGS},
        description={Red global de estaciones terrenas de código abierto para comunicaciones satelitales, que permite a operadores aficionados y profesionales compartir infraestructura de seguimiento y recepción.}
}

%Hamlib
\newglossaryentry{hamlib}{
        name={Hamlib},
        description={Biblioteca de código abierto que proporciona una interfaz estándar para controlar equipos de radioaficionados, incluyendo transceptores y rotores de antena.}
}

% GSaaS
\newglossaryentry{gsaas}{
        name={GSaaS},
        description={Ver \gls{groundstationasaservice}.}
}

\newglossaryentry{groundstationasaservice}{
        name={\textit{Ground Station as a Service}},
        description={Modelo de servicio que permite a operadores satelitales acceder a estaciones terrenas distribuidas geográficamente sin necesidad de poseer infraestructura propia, pagando por tiempo de uso.}
}

% Soapysdr
\newglossaryentry{soapysdr}{
        name={SoapySDR},
        description={Biblioteca de código abierto que proporciona una API unificada para interactuar con diferentes dispositivos de radio definida por software, permitiendo abstraer las diferencias entre fabricantes.}
}

% ZeroMQ
\newglossaryentry{zeromq}{
        name={ZeroMQ},
        description={Biblioteca de mensajería asincrónica de alto rendimiento que proporciona colas de mensajes sin necesidad de un broker dedicado, facilitando la comunicación entre procesos.}
}

% MQTT
\newglossaryentry{mqtt}{
        name={MQTT},
        description={Protocolo de mensajería ligero basado en publicación/suscripción, diseñado para comunicaciones de máquina a máquina (M2M) en redes con ancho de banda limitado o conexiones no confiables.}
}

% HDLC
\newglossaryentry{HDLC}{
        name={HDLC},
        description={Protocolo de capa de enlace de datos (High-Level Data Link Control) que estructura el flujo de bits en tramas delimitadas, incluyendo mecanismos de detección de errores mediante CRC y transparencia mediante bit stuffing.}
}

% SGP4
\newglossaryentry{sgp4}{
        name={SGP4},
        description={Modelo de propagación orbital simplificado (\textit{Simplified General Perturbations 4}) utilizado para calcular la posición y velocidad de objetos en órbita terrestre baja a partir de elementos orbitales \gls{tle}.}
}

% Skyfield
\newglossaryentry{skyfield}{
        name={Skyfield},
        description={Biblioteca de Python para cálculos astronómicos de alta precisión, incluyendo propagación orbital, transformaciones de marcos de referencia y correcciones como el \gls{efectodoppler}.}
}

% Efecto Doppler
\newglossaryentry{efectodoppler}{
        name={Doppler},
        description={Cambio aparente en la frecuencia de una onda debido al movimiento relativo entre la fuente emisora y el receptor. En comunicaciones satelitales, requiere corrección para compensar el desplazamiento de frecuencia causado por la velocidad orbital.}
}
