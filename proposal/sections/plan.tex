\subsection*{Proceso de desarrollo de software}

El desarrollo del software para la estación terrena se organizará en torno a una metodología ágil adaptada al contexto académico y a la naturaleza del proyecto. Se utilizará una variante de \textit{Scrum}, con iteraciones continuas, revisiones periódicas y entregas incrementales, lo que permite una mejora constante del producto y flexibilidad frente a cambios en los requisitos.

\subsection*{Metodología de trabajo}

El trabajo se dividirá en sprints de corta duración (longitud exacta a determinar), ajustables en función de la carga académica y el avance del proyecto. Cada sprint incluirá las siguientes etapas:

\begin{itemize}
    \item Planificación del sprint.
    \item Desarrollo de funcionalidades.
    \item Revisión del avance.
    \item Documentación técnica y funcional.
\end{itemize}

Al final de cada iteración se realizará una reunión de seguimiento para evaluar avances, detectar obstáculos y ajustar estimaciones. También se mantendrán reuniones periódicas con los tutores y los interesados del proyecto para asegurar la alineación con los objetivos generales.

Se buscará obtener un producto mínimo viable (MVP) en etapas tempranas, que cumpla con los requerimientos más críticos del sistema. A partir de allí, el desarrollo continuará de forma incremental hasta alcanzar la solución completa.

\subsection*{Gestión de alcance, tiempo y calidad}

El alcance del proyecto se define como la entrega de un sistema funcional que permita la recepción, procesamiento y envío de datos entre la estación terrena y el satélite, integrando una interfaz de usuario y funciones de control remoto. Se gestionará a través de una lista priorizada de funcionalidades, organizadas en entregables parciales.

Los tiempos se estimarán al inicio de cada sprint y se ajustarán iterativamente en función del desempeño actual. Como indicadores de avance se utilizarán: número de funcionalidades completadas, cumplimiento de hitos, cobertura de pruebas, porcentaje de documentación técnica y cantidad de errores detectados y resueltos.

La calidad del software se garantizará mediante revisiones de código entre pares, pruebas automatizadas, validaciones funcionales y documentación exhaustiva. Además, cada funcionalidad entregada deberá cumplir con criterios de aceptación previamente definidos.

\subsection*{Gestión de riesgos}

Se identificaron posibles riesgos técnicos y organizativos, mencionados en la sección anterior. Estos riesgos serán abordados mediante pruebas tempranas de viabilidad, prototipos rápidos y definición de soluciones alternativas. Se llevará un registro actualizado de los riesgos y su tratamiento.

\subsection*{Gestión de cambios}

Los cambios en los requisitos o en la planificación serán evaluados durante las reuniones de revisión de sprint. En caso de ser aceptados, se actualizará el backlog del proyecto y se ajustará la planificación futura.

\subsection*{Documentación}

El proyecto mantendrá una documentación completa y organizada, se tendrá:

\begin{itemize}
    \item Documentación funcional y de usuario: descripción de módulos, flujos, funcionalidades y uso general.
    \item Documentación técnica del sistema: estructura de carpetas, API, arquitectura, protocolos, interfaz con \textit{SDR}, etc.
    \item Minutas de reuniones internas y con tutores.
    \item Manual de instalación, despliegue y operación del sistema.
\end{itemize}

Toda la documentación se almacenará en el repositorio compartido y versionado, junto con el código fuente.

\subsection*{Hitos de avance}

\begin{enumerate}
    \item Análisis de requisitos y diseño general del sistema.
    \item Primer prototipo con recepción de señales desde \textit{SDR}.
    \item Implementación inicial del protocolo de comunicaciones.
    \item Transmisión y control remoto básico.
    \item Pruebas integradas con flujos de datos simulados.
    \item Desarrollo de la interfaz de usuario.
    \item Validación funcional y entrega final.
\end{enumerate}

El producto final incluirá el código fuente del sistema, su documentación técnica y funcional completa, scripts de instalación, manual de usuario y un conjunto de pruebas automatizadas y manuales para su verificación y validación.